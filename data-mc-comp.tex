 \documentclass[11pt]{article}
  \usepackage{rotating}
  \usepackage{pdflscape}
  \usepackage{graphicx}
    \textwidth 7.0in
    \textheight 9.5in
    \topmargin -1.0in
    \linewidth 7.0in
    \oddsidemargin -0.3in
    \evensidemargin -0.3in
        \begin{document}


 %%==================================================


        \tableofcontents

 \pagebreak


%\setcounter{section}{-1}

\section{ Description }

2021-10-16 \\

\vspace{1cm}

The only event selection that I added, in addition to what's in the H1 ntuple making code,
was to require $Q^2(e) > 220$~GeV$^2$ for both data and MC (Rapgap), where I chose to use
the electron method evaluation of $Q^2$.

The ntuples are in this directory: {\tt /data/owen/DIS-reco/h1-2021-10-14-v5f} \\
and the files are \\
{\tt dnn-output-h1-v2-Data.root} and {\tt dnn-output-h1-v2-Rapgap.root}.


\section{ Event distributions }

\includegraphics[width=0.98\linewidth]{plots-data-mc-comp/event-vars.pdf} \\
\includegraphics[width=0.98\linewidth]{plots-data-mc-comp/sigma-dphi.pdf}



 %%==================================================
\section{ DNN inputs }

\subsection{ Electron vars }
\includegraphics[width=0.98\linewidth]{plots-data-mc-comp/dnn-inputs-electron-vars.pdf}

\subsection{ HFS vars }
\includegraphics[width=0.98\linewidth]{plots-data-mc-comp/dnn-inputs-hfs-vars.pdf}

\subsection{ ISR vars }
\includegraphics[width=0.98\linewidth]{plots-data-mc-comp/dnn-inputs-isr-vars.pdf}

\subsection{ FSR vars }
\includegraphics[width=0.98\linewidth]{plots-data-mc-comp/dnn-inputs-fsr-vars.pdf}


\pagebreak

 %%==================================================
\section{ DNN outputs }

For each set, I have two 2D hists for MC and one for data.
The MC hists are DNN vs gen and DNN vs obs, where obs is
the value from one of the standard methods.  The data
hist is only the second type (DNN vs obs.).
The idea is the obs is a proxy for the gen value, so that you can
compare a 2D plot in MC with the same thing in data, where
the plot is close to what you want (DNN vs gen).

\subsection{ $Q^2$ }
\includegraphics[width=0.98\linewidth]{plots-data-mc-comp/dis-Q2-2d.pdf} \\
\includegraphics[width=0.98\linewidth]{plots-data-mc-comp/dis-log10Q2-2d.pdf}

\subsection{ $y$ }
\includegraphics[width=0.98\linewidth]{plots-data-mc-comp/dis-y-2d.pdf} \\
\includegraphics[width=0.98\linewidth]{plots-data-mc-comp/dis-log10y-2d.pdf}

\subsection{ $x$ }
\includegraphics[width=0.98\linewidth]{plots-data-mc-comp/dis-x-2d.pdf} \\
\includegraphics[width=0.98\linewidth]{plots-data-mc-comp/dis-log10x-2d.pdf}


 %%==================================================
\section{ Ratio histograms of $Q^2$, $y$, and $x$ }

\includegraphics[width=0.98\linewidth]{plots-data-mc-comp/Q2yx-ratio-hists-5ybins.pdf} \\



\end{document}

